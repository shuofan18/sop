\documentclass[12pt,]{article}
\usepackage{lmodern}
\usepackage{amssymb,amsmath}
\usepackage{ifxetex,ifluatex}
\usepackage{fixltx2e} % provides \textsubscript
\ifnum 0\ifxetex 1\fi\ifluatex 1\fi=0 % if pdftex
  \usepackage[T1]{fontenc}
  \usepackage[utf8]{inputenc}
\else % if luatex or xelatex
  \ifxetex
    \usepackage{mathspec}
    \usepackage{xltxtra,xunicode}
  \else
    \usepackage{fontspec}
  \fi
  \defaultfontfeatures{Mapping=tex-text,Scale=MatchLowercase}
  \newcommand{\euro}{€}
    \setmainfont{Times New Roman}
\fi
% use upquote if available, for straight quotes in verbatim environments
\IfFileExists{upquote.sty}{\usepackage{upquote}}{}
% use microtype if available
\IfFileExists{microtype.sty}{%
\usepackage{microtype}
\UseMicrotypeSet[protrusion]{basicmath} % disable protrusion for tt fonts
}{}
\usepackage[margin = 1in]{geometry}
\ifxetex
  \usepackage[setpagesize=false, % page size defined by xetex
              unicode=false, % unicode breaks when used with xetex
              xetex]{hyperref}
\else
  \usepackage[unicode=true]{hyperref}
\fi
\hypersetup{breaklinks=true,
            bookmarks=true,
            pdfauthor={Shuofan Zhang, Statistics},
            pdftitle={Statement of Purpose},
            colorlinks=true,
            citecolor=blue,
            urlcolor=blue,
            linkcolor=magenta,
            pdfborder={0 0 0}}
\urlstyle{same}  % don't use monospace font for urls
\setlength{\parindent}{0pt}
\setlength{\parskip}{6pt plus 2pt minus 1pt}
\setlength{\emergencystretch}{3em}  % prevent overfull lines
\setcounter{secnumdepth}{0}

%%% Use protect on footnotes to avoid problems with footnotes in titles
\let\rmarkdownfootnote\footnote%
\def\footnote{\protect\rmarkdownfootnote}

%% here
\usepackage{titling}
\setlength{\droptitle}{-8em}
\title{Statement of Purpose}
\pretitle{\begin{center}\LARGE}
%\posttitle{\par\end{center}\vspace{-6em}}

\author{Shuofan Zhang, Statistics}
\preauthor{\begin{center}
\large \lineskip 0.5em%
\begin{tabular}[t]{c}}
\postauthor{\end{tabular}\par\end{center}\vspace{-4em}}

%% here

  \date{}
  
\usepackage{setspace}\onehalfspacing

\begin{document}

\maketitle


It is my desire to pursue a Ph.D.~in Statistics at Rice University as
part of my long-term professional goal of becoming an academic
researcher. I have a strong interest in data visualization and data
mining.

It was the first statistics course of my master's program taught by
Professor Dianne Cook that piqued my interest. Part of the assessments
was an interesting competition on Kaggle\footnote{Kaggle is an online
  community of data scientists, owned by Google, Inc. See
  \url{https://www.kaggle.com/}} predicting housing prices. My random
forest regression outperformed all my classmates' linear regressions in
terms of out-of-sample mean squared error. Studying more on the machine
learning technique, I realized its similarity to estimation problems in
econometrics and became curious about its possible applications to this
field. I decided to pursue my interest in applying statistical
methodologies and transferred my specialization from actuarial studies
to applied econometrics.

In the year that followed, I received the \emph{Monash Business School
Student Excellence Award} for achieving the highest mark in seven of my
eight courses. In April 2018, I was chosen as one of the four
representatives of Monash University to participate in the
\emph{Econometric Game}\footnote{The Econometric Game is hosted by the
  University of Amsterdam. See \url{http://econometricgame.nl/}}, where
teams of postgraduate students examine a research topic and deliver
academic papers for competition. Thirty prestigious universities were
represented, including Harvard University and the University of
Cambridge. The research topic considered the detrimental effects of an
individual's unemployment on that individual's happiness, as well as on
a group's wellbeing. The dataset contained more than one thousand
variables. We first decided to choose the explanatory variables based on
empirical results from relevant literature, but realized this approach
was inefficient and may omit important information. Hence, I suggested
the Least Absolute Shrinkage and Selection Operator (LASSO) to conduct
variable selection, with the tuning parameter \(\lambda\) chosen by
cross validation. With the variables filtered, our team constructed an
ordered probit model with the raw responses to the survey item capturing
overall life satisfaction as the dependent variable. Under the
assumption of the homogeneous spillover effects amongst individuals in a
group, we estimated the multiplier between the effects on an individual
and a group. This great experience significantly stimulated my interest
in research and helped me understand the power of statistics.

As an attempt to explore the potential of statistical techniques, my
master's thesis employed deep learning to facilitate the hypothesis test
design, which was supervised by Professor Dianne Cook. The derivation of
hypothesis tests and their asymptotic distributions constitute a
considerable part of the statistics literature. However, the derivation
is often complex and the resulting test may lack power. For example, the
commonly used unit root tests in time series all suffer from low power
in distinguishing the unit root null from stationary alternatives. In my
thesis, I trained a binary deep learning classifier to test the null of
no structure against linear patterns in a scatter plot, as an
alternative to the conventional \(t\)-test. Tested on a large unseen
dataset, the power (\(1-\beta\)) of the classifier was always close to
the \(t\)-test holding the type I error (\(\alpha\)) constant. Given
that the \(t\)-test is known as the uniformly most powerful test under
such experimental settings according to the Neyman--Pearson lemma, this
finding implies that the deep learning model has the potential to
approach the unknown best test in more complicated situations. The study
was then extended to test the null of homoscedasticity against
heteroscedasticity using the re-trained classifier. A small dataset of
human evaluations was collected using the ``lineup'' protocol (Majumder,
Hofmann, and Cook 2013) and a specific form of the White test was
applied to provide a reference level of the test accuracy. The
classifier achieved much higher accuracy than both the White test and
the human evaluations. Using the data plot as the test statistic instead
of a single quantity, this approach could avoid complex derivations,
while exploiting the useful information in the plots (see Anscombe
(1973) for examples). Although the training design could be challenging,
these results encourage future research. For instance, by replacing the
scatter plot with time plots one can use this approach to perform a unit
root test. I am currently writing a first-author paper on these results
to submit to \emph{Statistical Analysis and Data Mining}. This study was
also presented by Professor Cook as the \(50^{th}\) \emph{Belz} Lecture
for the \emph{Statistical Society of Victoria}\footnote{The Victorian
  Branch of the Statistical Society of Australia Inc. See
  \url{https://www.statsoc.org.au/branches/victoria/}}.

After completing my master's degree, I was excited to accept an offer
from Professor Heather Anderson and Professor Farshid Vahid as a
research assistant to work on extensions to a paper studying
high-dimensional predictive regression with the LASSO (Koo et al. 2016).
In this project, I reviewed the literature to investigate the
compatibility (or restricted eigenvalue) condition and its implication
of choosing the tuning parameter \(\lambda\) for the \(\ell_1\) norm to
achieve the prediction optimality, while taking into account the
potential consequences of inconsistency for variable selection by the
LASSO (Bühlmann and Van De Geer 2011). I have been self-studying real
analysis by reading the book \emph{Principles of Mathematical Analysis}
(Rudin 1976) to better understand the relevant concepts. Comparison
between the LASSO, the adaptive LASSO and the group LASSO is under
consideration. In addition, the out-of-sample mean squared errors of
forecasting GDP growth and inflation rate using the LASSO on 146
macroeconomics variables are compared against other approaches including
an autoregressive model and a principal component analysis. The
potential co-integrating relationships in the selected variables are
being studied. Meanwhile, I am co-authoring a paper with the Learning
and Teaching team at Monash University which measures student levels of
perceptions of live-streaming, a new technology implemented in the
lectures. Our study adapted the CRiSP\footnote{CRiSP is the name of
  classroom response system perceptions questionnaire. (Richardson et
  al. 2015)} questionnaire which was validated by a combination of
factor analyses. Our results revealed three reliable scales: acceptance,
usability, and confidence. Following the validation results, I
investigated the correlations between the three scales and the
self-reported study attitudes using the estimated factor scores.

Although I am open to a variety of topics in statistics, there are
several professors at Rice University whose projects are especially
appealing to me: Professors Scott, Merényi and Allen. After reading some
of their papers, I believe their work is closely aligned with my skills
and interests and that Rice will be a great environment for me to
thrive.

\section*{References}\label{references}
\addcontentsline{toc}{section}{References}

\hypertarget{refs}{}
\hypertarget{ref-ANS73}{}
Anscombe, FJ. 1973. ``Graphs in Statistical Analysis.'' \emph{The
American Statistician} 27 (1): 17--21.

\hypertarget{ref-buhlmann}{}
Bühlmann, Peter, and Sara Van De Geer. 2011. \emph{Statistics for
High-Dimensional Data: Methods, Theory and Applications}. Springer
Science \& Business Media.

\hypertarget{ref-koo}{}
Koo, Bonsoo, Heather M Anderson, Myung Hwan Seo, and Wenying Yao. 2016.
``High-Dimensional Predictive Regression in the Presence of
Cointegration.'' \url{https://ssrn.com/abstract=2851677}.

\hypertarget{ref-MM13}{}
Majumder, Mahbubul, Heike Hofmann, and Cook Dianne. 2013. ``Validation
of Visual Statistical Inference, Applied to Linear Models.''
\emph{Journal of the American Statistical Association} 108 (503). Taylor
\& Francis Group: 942--56.

\hypertarget{ref-rudin}{}
Rudin, Walter. 1976. ``Principles of Mathematical Analysis
(International Series in Pure \& Applied Mathematics).'' McGraw-Hill
Publishing Co.

\end{document}
