\documentclass[12pt,]{article}
\usepackage{lmodern}
\usepackage{amssymb,amsmath}
\usepackage{ifxetex,ifluatex}
\usepackage{fixltx2e} % provides \textsubscript
\ifnum 0\ifxetex 1\fi\ifluatex 1\fi=0 % if pdftex
  \usepackage[T1]{fontenc}
  \usepackage[utf8]{inputenc}
\else % if luatex or xelatex
  \ifxetex
    \usepackage{mathspec}
    \usepackage{xltxtra,xunicode}
  \else
    \usepackage{fontspec}
  \fi
  \defaultfontfeatures{Mapping=tex-text,Scale=MatchLowercase}
  \newcommand{\euro}{€}
    \setmainfont{Times New Roman}
\fi
% use upquote if available, for straight quotes in verbatim environments
\IfFileExists{upquote.sty}{\usepackage{upquote}}{}
% use microtype if available
\IfFileExists{microtype.sty}{%
\usepackage{microtype}
\UseMicrotypeSet[protrusion]{basicmath} % disable protrusion for tt fonts
}{}
\usepackage[margin = 1in]{geometry}
\ifxetex
  \usepackage[setpagesize=false, % page size defined by xetex
              unicode=false, % unicode breaks when used with xetex
              xetex]{hyperref}
\else
  \usepackage[unicode=true]{hyperref}
\fi
\hypersetup{breaklinks=true,
            bookmarks=true,
            pdfauthor={Shuofan Zhang, Economics Department},
            pdftitle={Statement of Purpose},
            colorlinks=true,
            citecolor=blue,
            urlcolor=blue,
            linkcolor=magenta,
            pdfborder={0 0 0}}
\urlstyle{same}  % don't use monospace font for urls
\setlength{\parindent}{0pt}
\setlength{\parskip}{6pt plus 2pt minus 1pt}
\setlength{\emergencystretch}{3em}  % prevent overfull lines
\setcounter{secnumdepth}{0}

%%% Use protect on footnotes to avoid problems with footnotes in titles
\let\rmarkdownfootnote\footnote%
\def\footnote{\protect\rmarkdownfootnote}


  \title{Statement of Purpose}
    \author{Shuofan Zhang, Economics Department}
    \date{}
  
\usepackage{setspace}\onehalfspacing

\begin{document}

\maketitle


It is my desire to pursue a PhD in Economics at University of
California, San Diego as part of my long-term professional goal of
becoming an academic researcher. I have a strong interest in
econometrics and macroeconomics.

This interest became a passion from the first econometric course of my
master's program that was taught by Professor Farshid Vahid. I was
fascinated by the idea of tackling real-world problems based on economic
theories and statistical methods, I could not stop reading the textbook
written by Wooldridge (2016). In my reading I discovered three errors in
the equations. Surprised by these errors, Professor Vahid wrote to
Professor Jeffrey Wooldridge, which resulted in his appreciation, public
praise in class, and extra course credit points. Encouraged by this
experience, I decided to pursue my interest in econometrics and
transferred my specialization from actuarial studies to applied
econometrics.

In the year that followed, I received the \emph{Monash Business School
Student Excellence Award} for achieving the highest mark in seven of my
eight courses. In April 2018, I was chosen as one of the four
representatives of Monash University to participate in the
\emph{Econometric Game}\footnote{The Econometric Game is hosted by the
  University of Amsterdam. See \url{http://econometricgame.nl/}}, where
teams of postgraduate students examine a research topic and deliver
academic papers for competition. Thirty prestigious universities were
represented, including Harvard University and University of Cambridge.
The research topic considered the detrimental effects of an individual's
unemployment on that individual's happiness, as well as on a group's
wellbeing. After reviewing the relevant literature for the empirical
support on selecting explanatory variables, our team constructed an
ordered probit model with the raw responses to the survey item capturing
overall life satisfaction as the dependent variable. Under the
assumption of the homogeneous spillover effects amongst individuals in a
group, we then estimated the multiplier between the effects on an
individual and a group. The potential simultaneity bias and implications
were discussed. This great experience of conducting rigorous research
provided me with important new insights into how unemployment affects
well-being, and significantly stimulated my interest in empirical
research in economics.

Measures of happiness, previously tools used by psychologists, now are
widely adopted by economists; likewise, statistical advances are
infiltrating econometrics.

As an attempt to introduce a statistical technique to econometrics, my
master's thesis employed deep learning to facilitate the hypothesis test
design, which was supervised by Professor Dianne Cook. Deriving
hypothesis tests and their asymptotic distributions constitute a
considerable part of the econometric literature. However, the derivation
is often complex and the resulting test may lack power. For example,
(the) Hausman tests and (the) unit root tests all suffer from low power
when certain assumptions are violated. In my thesis, I trained a binary
deep learning classifier to test the null of no structure against linear
patterns in a scatter plot, as an alternative to the conventional
\(t\)-test. Tested on a large unseen dataset, the power (\(1-\beta\)) of
the classifier was always close to the \(t\)-test holding the type I
error (\(\alpha\)) constant. Given that the \(t\)-test is known as the
uniformly most powerful test under such experimental settings according
to the Neyman--Pearson lemma, this finding suggests that the deep
learning model has a potential to approach the unknown best test in more
complicated situations. The study was then extended to test the null of
homoscedasticity against heteroscedasticity using the re-trained
classifier. A small dataset of human evaluations was collected via an
online questionnaire and a specific form of the White test was applied
to provide a reference level of the test accuracy. The classifier
achieved much higher accuracy than both the White test and the human
evaluations. Using the data plot as the test statistic instead of a
single quantity, this approach could avoid complex derivations, while
exploiting the useful information in the plots (see Anscombe (1973) for
examples). Although the training design could be challenging, these
results encourage future research. For instance, by replacing the
scatter plot with time plots one can use this approach to test the unit
root in time series. I am currently writing a first-author paper on
these results to submit to the \emph{Statistical Analysis and Data
Mining}. This study was also presented by Professor Cook as the
\(50^{th}\) \emph{Belz} Lecture for the \emph{Statistical Society of
Victoria}\footnote{The Victorian Branch of the Statistical Society of
  Australia Inc. See \url{https://www.statsoc.org.au/branches/victoria/}}.

After completing my master's degree, I was excited to accept an offer
from Professor Heather Anderson and Professor Farshid Vahid as a
research assistant to work on a paper studying the high-dimensional
predictive regression with the Least Absolute Shrinkage and Selection
Operator (LASSO) (Koo et al. 2016). In this project, I reviewed the
literature to investigate the compatibility (or restricted eigenvalue)
condition and its implication of choosing the tuning parameter
\(\lambda\) for the \(\ell_1\) norm to achieve the prediction
optimality, while taking into account the potential consequences of
inconsistency for variable selection by the LASSO (Bühlmann and Van De
Geer 2011). I have been self-studying real analysis by reading the book
\emph{Principles of Mathematical Analysis} (Rudin 1976) to better
understand the relevant concepts. Comparison between the LASSO, the
adaptive LASSO and the group LASSO is under consideration. In addition,
the out-of-sample mean squared errors of forecasting GDP growth and
inflation rate using the LASSO on 146 macroeconomics variables are
compared against other approaches including an autoregressive model and
a principal component analysis. The potential co-integrating
relationships in the selected variables are being studied. Observing the
series chosen by this data-driven methodology and seeking possible
economic theory-based explanations are intriguing and provoked my
intellectual curiosity. I desire to continue to work on macroeconomic
problems and apply cutting-edge techniques to challenging areas such as
high-dimensional data and forecasting economic or financial series.

Meanwhile, I am co-authoring a paper with the Learning and Teaching team
at Monash University which measures student levels of perceptions of
live-streaming, a new technology implemented in the lectures. Our study
adapted the CRiSP\footnote{CRiSP is the name of classroom response
  system perceptions questionnaire. (Richardson et al. 2015)}
questionnaire which was validated by a combination of factor analyses.
Our results revealed three reliable scales: acceptance, usability, and
confidence. Following the validation results, I investigated the
correlations between the three scales and the self-reported study
attitudes using the estimated factor scores. The causal effect of
live-streaming on academic performance is also being studied using a
fixed effects model with panel data accounting for unobserved
heterogeneity.

Though I am open to a variety of research in economics, I find the work
of Professors Elliott, Timmermann, and Hamilton are especially
interesting to me. I am confident that my talent in mathematics combined
with my various research experiences in economics will allow me to
contribute positively to the PhD program at University of California,
San Diego.

\section*{References}\label{references}
\addcontentsline{toc}{section}{References}

\hypertarget{refs}{}
\hypertarget{ref-ANS73}{}
Anscombe, FJ. 1973. ``Graphs in Statistical Analysis.'' \emph{The
American Statistician} 27 (1): 17--21.

\hypertarget{ref-buhlmann}{}
Bühlmann, Peter, and Sara Van De Geer. 2011. \emph{Statistics for
High-Dimensional Data: Methods, Theory and Applications}. Springer
Science \& Business Media.

\hypertarget{ref-koo}{}
Koo, Bonsoo, Heather M Anderson, Myung Hwan Seo, and Wenying Yao. 2016.
``High-Dimensional Predictive Regression in the Presence of
Cointegration.''

\hypertarget{ref-rudin}{}
Rudin, Walter. 1976. ``Principles of Mathematical Analysis
(International Series in Pure \& Applied Mathematics).'' McGraw-Hill
Publishing Co.

\hypertarget{ref-wooldridge2016}{}
Wooldridge, Jeffrey M. 2016. \emph{Introductory Econometrics: A Modern
Approach, 6th Edition}. Cengage.

\end{document}
